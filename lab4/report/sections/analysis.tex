%Analyze the specifications and requirements.

In this assignment, we will implement a signal scaler.
A scaler is used to convert between sample rates.

Specifcally, we will implement a scaler that will upscape from a sample rate of 44.1kHz to a sample rate of 48kHz.

Scalers can be implemented in multiple ways.
First of all, we can implement a scaler directly, which does the entire transformation in one component.
Another option is to chain together multiple smaller scalers to achieve the same end result.

Inside each scaler, we have a choice between directly implementing the equation for the scaler as a function, or using a composite design that consists of an upscaler, a low-pass filter and a downscaler.

Effectively, we will be scaling with ratio $\frac{160}{147}$.
In the composite design, that means first upscaling with factor 160, then applying the filter and finally downscaling with factor 147.

The coefficients that the low-pass filter takes as its input are the output of the \texttt{lanczos2} function.
Since these values are constant, we will pre-compute them.
When we program the FPGA, we will load the values into ROM.

Specifically, our requirements are:

\begin{itemize}
	\item The sample rate of the output must be $\frac{160}{147}$ times the sample rate of the input.
	\item The output must be a resampled version of the input (meaning that they should sound the same).
	\item The filter must support an input sample rate of at least 44.1kHz.
	\item To run on the board, we need to make sure that the filter we design is able to run with a clock frequency of at least 100MHz.
	\item The top-level Verilog module of our implementation must be called \texttt{filter} to work with the rest of the environment.
\end{itemize}

The filter may produce start-up noise.
