%Present/motivate key ideas/decisions, design options, alternatives, trade-offs.
%Draw architecture block diagram (= picture!).

Figure~\ref{fig:design:block} gives the block diagram for the design of our scaler from assignment 4.

\begin{figure}[h]
	\centering
	\def\svgwidth{0.6\textwidth}
	\input{../../lab4/report/imgs/block.pdf_tex}
	\caption{Block diagram for the single scaler}
	\label{fig:design:block}
\end{figure}

Our design is based on the same formula used in assignment 4.
However, using a similar implementation would severely limit the amount of streams the program would be able to handle.
As a result, as described in the assignment, we opted to increase parallel processing power at the cost of resource usage.

We use four block RAMs, each with $n$ entries, to store the last four values for each stream, as they are required by the formula.
The way the testbench and the program are set up, we need to produce an output every clock cycle. 
However, calculating the entire output in a single clock cycle would make the program far too slow.
To solve this, we divide the calculation into two parts.

First, we multiply each of the four values with the corresponding coefficient value in parallel.
Secondly, the results of these multiplications are added to each other and presented as output.

Note that these steps still occur in the same clock cycle, in parallel.
This is possible, as they are for different streams.
The result of the multiplication steps are used for the sum step in the next cycle.
So at any point, two streams are being calculated at the same time.
Doing these calculations in parallel makes for a much faster program, as multiplications tend to be expensive.

The downside, however, is that the program will not work for a single stream.

After every 'round`, where the full range of streams has been handled a single time, we compute whether or not we need to request an input value for the next round.
