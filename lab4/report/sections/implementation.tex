%Explain functional correctness of your Verilog programs (include your complete Verilog programs in an appendix).
%Explain #clock cycles per sample time Ts. Include waveforms.
%Report, analyze & explain FPGA-resource usage and utilization {#multipliers, #BRAMS, #LUTs} in relation to your design.
%Report, analyze & explain (min) sample time Ts and (max) sample frequency fs, both after synthesis and after placement & routing.
Appendix~\ref{app:source:filter} gives the Verilog implementation for the FIR filter.

\paragraph{}
To speed up our design, we made a few implementation choices that affect readability.
Most notably, we unwrapped the modulo operation into a code branch.
This allows us to get rid of some specialized hardware, which is relatively slow.

Furthermore, we split up the calculation step into two parts, as we did last project.

\paragraph{Resource usage}
Our resource usage, as given by the synthesis report, is as follows.

\begin{verbatim}
# RAMs                                                 : 2
 4x10-bit single-port Read Only RAM                    : 1
 640x16-bit single-port Read Only RAM                  : 1
# Multipliers                                          : 1
 16x16-bit multiplier                                  : 1
# Adders/Subtractors                                   : 4
 10-bit adder                                          : 1
 3-bit adder                                           : 1
 32-bit adder                                          : 1
 8-bit addsub                                          : 1
# Registers                                            : 9
 1-bit register                                        : 3
 16-bit register                                       : 1
 3-bit register                                        : 1
 32-bit register                                       : 2
 64-bit register                                       : 1
 8-bit register                                        : 1
# Comparators                                          : 2
 8-bit comparator greater                              : 1
 8-bit comparator lessequal                            : 1
# Multiplexers                                         : 18
 1-bit 2-to-1 multiplexer                              : 8
 16-bit 4-to-1 multiplexer                             : 1
 3-bit 2-to-1 multiplexer                              : 4
 32-bit 2-to-1 multiplexer                             : 4
 8-bit 2-to-1 multiplexer                              : 1
\end{verbatim}

The synthesis has modelled the ROM storage for the coefficients as read-only RAM.
We use one multiplier, in the computation step.
The adders are used in different places throughout the filter, during initialization, modulo calculation and the computation step.
We use as many registers we expect.
The comparators are used for the initialization for-loop and the modulo calculation.

The multiplexers are due to the way we access the wires.

The synthesis report gives us the following values.
\begin{description}
	\item[Minimum period] 10.411ns
	\item[Maximum frequency] 96.053MHz
\end{description}

The report after the placement \& routing step gives us the following values.
\begin{description}
	\item[Minimum period] 9.549ns
	\item[Maximum frequency] 104.723MHz
\end{description}

We note that in the previous projects, the synthesis report consistently gave us higher values for the maximum frequency than the post placement \& routing report.
Our explanation for this is that our design triggered some routing optimalizations which allowed for a higher maximum frequency, which were not obvious from just the hardware requirements and critical paths.
