%Explain functional correctness of your Verilog programs (include your complete Verilog programs in an appendix).
%Explain #clock cycles per sample time Ts. Include waveforms.
%Report, analyze & explain FPGA-resource usage and utilization {#multipliers, #BRAMS, #LUTs} in relation to your design.
%Report, analyze & explain (min) sample time Ts and (max) sample frequency fs, both after synthesis and after placement & routing.

\seqfilter
Appendix~\ref{app:source:seq} gives the Verilog implementation for the sequential FIR filter.

\paragraph{}
It took a lot of versions to get to an implementation of our design that was fast enough to run on the hardware.
We started at an implementation that ran at 85MHz.
When we tried to optimize that with the help of the instructor, it ran at 65MHz.
After a complete re-write (using the same principle design), we were able to achieve 104MHz.

\paragraph{Resource usage}
Our resource usage, as given by the synthesis report, is as follows.

\begin{verbatim}
# Multipliers                                          : 1
 16x16-bit multiplier                                  : 1
# Adders/Subtractors                                   : 2
 32-bit adder                                          : 1
 6-bit adder                                           : 1
# Registers                                            : 1097
 Flip-Flops                                            : 1097
# Shift Registers                                      : 16
 63-bit dynamic shift register                         : 16
# Multiplexers                                         : 11
 1-bit 2-to-1 multiplexer                              : 2
 32-bit 2-to-1 multiplexer                             : 6
 6-bit 2-to-1 multiplexer                              : 3
# Logic shifters                                       : 1
 1008-bit shifter logical left                         : 1
\end{verbatim}

As intended, we only use one multiplier.
The 32-bit adder is the one we use in the main computation step, the 6-bit adder is for calculating the \texttt{calc\_buf}.
The shift registers are used for initializing the \texttt{input\_buf}, as is the logic shifter.
We would expect to use around 620 registers, with another 64 for the multiplier and adder.
We are not sure what the other roughly 400 registers are used for.

The multiplexers are due to the way we access the wires.

The synthesis report gives us the following values.
\begin{description}
	\item[Minimum period] 8.186ns
	\item[Maximum frequency] 122.154MHz
\end{description}

The report after the placement \& routing step gives us the following values.
\begin{description}
	\item[Minimum period] 9.591ns
	\item[Maximum frequency] 104.264MHz
\end{description}

\strengthfilter
Appendix~\ref{app:source:strength} gives the Verilog implementation for the strength-reduced FIR filter.

The synthesis report gives us the following values.
\begin{description}
	\item[Minimum period] 8.186ns
	\item[Maximum frequency] 122.154MHz
\end{description}

The report after the placement \& routing step gives us the following values.
\begin{description}
	\item[Minimum period] 9.341ns
	\item[Maximum frequency] 107.055MHz
\end{description}
