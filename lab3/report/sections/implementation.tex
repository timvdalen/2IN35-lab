%Explain functional correctness of your Verilog programs (include your complete Verilog programs in an appendix).
%Explain #clock cycles per sample time Ts. Include waveforms.
%Report, analyze & explain FPGA-resource usage and utilization {#multipliers, #BRAMS, #LUTs} in relation to your design.
%Report, analyze & explain (min) sample time Ts and (max) sample frequency fs, both after synthesis and after placement & routing.

\seqfilter
Appendix~\ref{app:source:seq} gives the Verilog implementation for the sequential FIR filter.

\paragraph{}
Unfortunately, we were not able to get our filter to work at at least 100MHz.
Our fastest implementation ran at 85MHz.
The results below are from an implementation that we worked on with the instructor, trying to speed things up.

\paragraph{Resource usage}
Our resource usage, as given by the synthesis report, is as follows.

\begin{verbatim}
# Multipliers                                          : 1
 16x16-bit multiplier                                  : 1
# Adders/Subtractors                                   : 4
 32-bit adder                                          : 1
 7-bit adder                                           : 2
 8-bit adder                                           : 1
# Registers                                            : 9
 1-bit register                                        : 2
 1008-bit register                                     : 3
 16-bit register                                       : 1
 32-bit register                                       : 1
 7-bit register                                        : 1
 8-bit register                                        : 1
# Comparators                                          : 1
 7-bit comparator lessequal                            : 1
# Multiplexers                                         : 75
 16-bit 2-to-1 multiplexer                             : 63
 16-bit 63-to-1 multiplexer                            : 2
 32-bit 2-to-1 multiplexer                             : 3
 7-bit 2-to-1 multiplexer                              : 4
 8-bit 2-to-1 multiplexer                              : 3
\end{verbatim}

Though our design respects the limit of one multiplier, we are using a lot of additional hardware.
The 32-bit adder is the one we use in the main computation step, the 7-bit and 8-bit adders are used for increasing the pointers.
The $<$ comparator is used in the reset block.

We think the amount of multiplexers is due to the way we read data from the wires, but we are not sure.

The synthesis report gives us the following values.
\begin{description}
	\item[Minimum period] 12.251ns
	\item[Maximum frequency] 81.628MHz
\end{description}

The report after the placement \& routing step gives us the following values.
\begin{description}
	\item[Minimum period] 15.590ns
	\item[Maximum frequency] 64.144MHz
\end{description}

\strengthfilter
Appendix~\ref{app:source:strength} gives the Verilog implementation for the strength-reduced FIR filter.
